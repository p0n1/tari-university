## Rank-1 Constraint System with Application to Bulletproofs


- Introduction
- Arithmetic Circuits
- Rank-1 Constraint Systems
- From Arithmetic Circuits to Programmable Constraint Systems for Bulletproofs 
- Interstellar's Bulletproof Constraint System
- Conclusion

---

## Introduction

- Originally proposed topic description: 
  - Fast and efficient transaction validation is one of the main reasons for Layer 2 Scaling, especially for blockchains using confidential transactions.  
  - Rank-1 Constraint Systems (R1CS) are used to achieve faster zero-knowledge proofs, particularly in zk-SNARKs and Bulletproofs. 
  - An R1CS is “basically a language for arithmetic circuits with bilinear gates”. It defines a set of bilinear equations that serve as constraints suitable for ZK proofs. 
  - Tentatively, the report will be split into two parts; 
    PART I    :  R1CS in zk-SNARKs  
    PART II   :  R1CS in Bulletproofs 

--- 

## Introduction

- From the research that has gone into this topic, came out 

(a) A curated content on zkSNARKs, 
(b) R1CS with Applications to Bulletproofs  

So, zkSNARKs have not yet been fully covered in the TLU. 
A fully fledged report on zkSNARKs will be considered later on. 

--- 

## Introduction

Bulletproofs on the other hand have been extensively covered. 

#### ![bulletproofs-r1cs-pic](https://github.com/tari-labs/tari-university/blob/master/src/cryptography/r1cs-bulletproofs/sources/bulletproofs-r1cs-pic.png)

 
--- 

## Introduction 

R1CS form a thin slice of the TLU reports on Bulletproofs. 
The big challenge was to avoid rehearsing things already presented under Bulletproofs. 

#### ![r1cs-zoomin](https://github.com/tari-labs/tari-university/blob/master/src/cryptography/r1cs-bulletproofs/sources/r1cs-zoomin.png)


--- 

## Introduction

So this talk will focus specifically on; 
- What an R1CS is 
- The link between Arithmetic Circuits and R1CS 
- Comparing ZK Proofs for Arithmetic Circuits (Bootle et al.) and Programmable Constraints Systems (Cathie Yun et al.)

--- 

## Introduction

- We are not worried too much about specific fields. 
   - A field is a set of objects (numbers) on which one can define some “addition” and “multiplication” 

   -  The idea of R1CS is not that sophisticated or complex. 
   -  The main issue here is language. It’s about expressing computational problems in a way that enables fast computations.


---

## Arithmetic Circuits

-  Many problems in Cryptography are expressible as polynomials. Hence Arithmetic Circuits. 
-  An arithmetic circuit is a directed acyclic graph. 
- The vertices of are called gates, while the edges are called wires. 
- Every other gate in an Arithmetic Circuit is labeled with either  $\oplus$  or  $\otimes$  indicating an “addition gate” or “multiplication gate”, respectively.

Example:

#### ![basic-multiplication-gate](https://github.com/tari-labs/tari-university/blob/master/src/cryptography/r1cs-bulletproofs/sources/basic-multiplication-gate.png)

- As an equation; that’s  
$ a_L \circ a_R = a_O  $ 
which is known as the Hadamard relation, and can be rewritten as
$a_L \circ a_R - a_O = 0 $

---

## Arithmetic Circuits

- Arithmetic Circuits are the most standard model for expressing computations in a low-level language.

#### ![polynomial-eg-ac](https://github.com/tari-labs/tari-university/blob/master/src/cryptography/r1cs-bulletproofs/sources/polynomial-eg-ac.png)

- The output of the above Arithmetic Circuit is the polynomial:   $x^2\_1 \cdot x\_2 + x\_1 + 1 $. 
- The above e.g. of an Arithmetic Circuit can be described by the following set of equations; 
$$
u = x_1 \cdot x_1 \quad \text{,} \quad v = u \cdot x_2 \quad \text{,} \quad y = x_1 + 1 \quad \text{and} \quad z = v + y
$$

---

## Constraint Systems

- Bootle et al. expressed arithmetic circuit satisfiability in terms of the Hadamard relation and linear constraints. 

$$
\mathbf{W_L\cdot { a_L} + W_R\cdot { a_R} + W_O\cdot { a_O } = c }
$$  

- Bunz et al. incorporated a vector  $\mathbf{v}$  and vector of weights  $\mathbf{W_V}$ to the Bootle et al. definition [[4]]:

$$
\mathbf{W_L\cdot { a_L} + W_R\cdot { a_R} + W_O\cdot { a_O } = W_V\cdot { v + c} }
$$

- where. $ \mathbf{v}$  is a secret vector of openings ${v_i}$ of the Pedersen Commitments  $V_i$  and 
$\mathbf{W_V}$  is a vector of weights for all commitments  $V_i$.


---

## Rank-1 Constraint Systems 

- An R1CS is a system that consists of two sets of constraints [[3]]: 

  - ${ n}$  multiplicative constraints,  $ \mathbf{ a_L \circ a_R = a_O } $,  and
  - ${ q}$  linear constraints,  $\mathbf{W_L\cdot { a_L} + W_R\cdot { a_R} + W_O\cdot { a_O } = W_V\cdot { v + c} } $. 

--- 

## Rank-1 Constraint Systems 

- The zkSNARK's corresponding definition of R1CS: 

  - A sequence of groups of three vectors ${ \bf{a_L}}, { \bf{a_R}}, { \bf{a_O}} ,$ and the 
solution to an R1CS is a vector ${ \bf{s}}$ that satisfies the equation:
$$
{ \langle {\mathbf{a_L} , \mathbf{s}} \rangle \cdot \langle {\mathbf{a_R} , \mathbf{s}} \rangle - \langle {\mathbf{a_O} , \mathbf{s}} \rangle = 0 }
$$
where
$$\langle {\mathbf{a_L} , \mathbf{s}} \rangle = a_{L, 1} \cdot s_1 + a_{L, 2} \cdot s_2 + \cdots + a_{L, n} \cdot s_n $$
which is the innerproduct of the vectors 
$ \mathbf{a\_{L}}  =  (a\_{L,1}, a\_{L,2}, ... , a\_{L,n} )$
and
$ {\mathbf{s}} = (s\_1, s\_2, ... , s\_n) $.

--- 

## R1CS Example

- A solution vector to the equation ${x^2_1 x_2 + x_1 + 1 = 22}$. can be formatted as 
${ { s = ( const , x_1 , x_2 , z , u , v , y )}}$, and in particular  ${ { s = ( 1 , 3 , 2 , 22 , 9 , 18 , 4 )}}$. 

<div align="center"><b>Table 1: Equations and Rank-1 Constraint System Vectors</b></div>  


| Equation                        | Rank-1 Constraint System Vectors                             |
| ------------------------------- | ------------------------------------------------------------ |
| ${ u = x_1\cdot x_1}$          | $ {\bf{a_L}} = ( 0 , 1 , 0 , 0 , 0 , 0 , 0 ) , \ \ {\bf{a_R}} = ( 0 , 1 , 0 , 0 , 0 , 0 , 0  ) ,\ \ {\bf{a_O}} = ( 0 , 0 , 0 , 0 , 1 , 0 , 0  ) $ |
| $ { v = u\cdot x_2 }$          | $ {\bf{a_L}} = ( 0 , 0 , 0 , 0 , 1 , 0 , 0 ) ,\ \ {\bf{a_R}} = ( 0 , 0 , 1 , 0 , 0 , 0 , 0  ),\ \ {\bf{a_O}} = ( 0 , 0 , 0 , 0 , 0 , 1 , 0 )  $ |
| $ { y = 1\cdot( x_1 + 1 ) } $ | ${\bf{a_L}} = ( 1 , 1 , 0 , 0 , 0 , 0 , 0 ),\ \ {\bf{a_R}} = ( 1 , 0 , 0 , 0 , 0 , 0 , 0 ),\ \ {\bf{a_O}} = ( 0 , 0 , 0 , 0 , 0 , 0 , 1 ) $ |
| $ { z = 1\cdot( v + y )} $    | ${\bf{a_L}} = ( 0 , 0 , 0 , 0 , 0 , 1 , 1 ),\ \ {\bf{a_R}} = ( 1 , 0 , 0 , 0 , 0 , 0 , 0 ),\ \ {\bf{a_O}} = ( 0 , 0 , 0 , 1 , 0 , 0 , 0 )$ |


--- 

## R1CS Example

- A solution vector to the equation ${x^2_1 x_2 + x_1 + 1 = 22}$. can be formatted as 
${ { s = ( const , x_1 , x_2 , z , u , v , y )}}$, and in particular  ${ { s = ( 1 , 3 , 2 , 22 , 9 , 18 , 4 )}}$. 


#### ![Table-1-r1cs-eg](https://github.com/tari-labs/tari-university/blob/master/src/cryptography/r1cs-bulletproofs/sources/Table-1-r1cs-eg.png)


--- 

## Arithmetic Circuits to Programmable Constraint Systems


- "Zero-knowledge Proofs for Arithmetic Circuits" by Bootle et al. 
- "Programmable Constraint Systems for Bulletproofs" by Cathie Yun, Interstellar. 
- "Bulletproofs: Short Proofs for Confidential Transactions and More" by Bunz et al. 

#### ![Table-2-comparison](https://github.com/tari-labs/tari-university/blob/master/src/cryptography/r1cs-bulletproofs/sources/Table-2-comparison.png)

--- 

## Interstellar's Bulletproof Constraint System 

- Dalek's constraint system, as defined earlier is 
a collection of arithmetic constraints of two types, 
   - multiplicative constraints and linear constraints, over a set of 
high-level and low-level variables.


- A constraint system can be build in two steps: 
      1. Committing to secret inputs and allocating high-level variables corresponding to the inputs.
      2. Selecting a suitable combination of multiplicative constraints and linear constraints, as well as requesting a random 
   scalar in response to the high-level variables already committed.  

--- 

## Interstellar's Bulletproof Constraint System 

- Lovesh Harchandani outlines ZK proofs that use Bulletproofs is follows: 

   - The prover commits to a value(s) that they want to prove knowledge of. 
   - The prover generates the proof by enforcing the constraints over the committed values and any additional public 
   values. The constraints might require the prover to commit to some additional variables. 
   - The Prover sends the verifier all the commitments made in step 1 and step 2 along with the proof from step 2. 
   - The verifier now verifies the proof by enforcing the same constraints over the commitments plus any public values.

--- 

## Interstellar's Bulletproof Constraint System 

- Factors Example:

#### ![factor-eg-allsteps](https://github.com/tari-labs/tari-university/blob/master/src/cryptography/r1cs-bulletproofs/sources/factor-eg-allsteps.png)

--- 

## Interstellar's Bulletproof Constraint System 

- Factors Example:

#### ![factor-eg-step5](https://github.com/tari-labs/tari-university/blob/master/src/cryptography/r1cs-bulletproofs/sources/factor-eg-step5.png)

--- 


## Interstellar's Bulletproof Constraint System 

- Factors Example:

#### ![factor-eg-step9](https://github.com/tari-labs/tari-university/blob/master/src/cryptography/r1cs-bulletproofs/sources/factor-eg-step9.png)

--- 

## Interstellar's Bulletproof Constraint System 

### About Gadgets

- Gadgets are building blocks of constraint systems; 
   - Examples are "shuffle", “merge”, “split” and a “range proof”. 

#### ![two-shuffle-pic](https://github.com/tari-labs/tari-university/blob/master/src/cryptography/r1cs-bulletproofs/sources/two-shuffle-pic.png)

--- 

## Interstellar's Bulletproof Constraint System 

### About Gadgets 
   - Gadgets are composable, thus forming a more complex gadget.

#### ![composable-gadgets-1](https://github.com/tari-labs/tari-university/blob/master/src/cryptography/r1cs-bulletproofs/sources/composable-gadgets-1.png)

---

## Interstellar's Bulletproof Constraint System 

### About Gadgets 
   - Gadgets are composable, thus forming a more complex gadget.

#### ![composable-gadgets-2](https://github.com/tari-labs/tari-university/blob/master/src/cryptography/r1cs-bulletproofs/sources/composable-gadgets-2.png)

--- 

## Interstellar's Bulletproof Constraint System 

#### ![two-shuffle-CYun](https://github.com/tari-labs/tari-university/blob/master/src/cryptography/r1cs-bulletproofs/sources/two-shuffle-CYun.png) 

--- 

## Interstellar's Bulletproof Constraint System 

#### ![prover-instance-CYun](https://github.com/tari-labs/tari-university/blob/master/src/cryptography/r1cs-bulletproofs/sources/prover-instance-CYun.png) 

--- 

## Interstellar's Bulletproof Constraint System 

#### ![verifier-instance-CYun](https://github.com/tari-labs/tari-university/blob/master/src/cryptography/r1cs-bulletproofs/sources/verifier-instance-CYun.png)

--- 

## Interstellar's Bulletproof Constraint System 

  - Perfomance 
  
#### ![bulletproof-performance](https://github.com/tari-labs/tari-university/blob/master/src/cryptography/r1cs-bulletproofs/sources/bulletproof-performance.png)  
  
   
--- 

## Conclusion

All-in-all, Yun regards constraint systems as "very powerful because they can represent any efficiently 
verifiable program." 



 
